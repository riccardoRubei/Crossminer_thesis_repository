\subsection{Work Description}
The purpose of this thesis is the implementation of two approaches, MUDABlue and Clan, with the aim of compare the results of new tool by CROSSMINER development team (CrossSim). CROSSSIM (Cross Project Relationships for Computing Open Source Software Similarity), is an approach that makes use of graphs for rep-resenting different kinds of relationships in the OSS ecosystem. In particular, with the adoption of the graph representation, we are able to transform the relationships among non-human artifacts, e.g. API utilizations, source code, interactions, and humans, e.g. developers into a mathematically computable format, i.e. one that
facilitates various types of computation techniques. Naturally this kind of approaches has to be evaluated, and confronted with others similar tools. My work helps addressing this challenge providing these two tools and evaluating all the results to show how nice is CrossSim.